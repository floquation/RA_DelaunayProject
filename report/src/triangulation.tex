\section{Delaunay triangulation}
\label{sec:triangulation}

There are 2 common algorithms to obtain a delaunay triangulation.
Both algorithms are incremental.

\subsection{Bowyer-Watson}
\label{sub:bowyer_watson}

Unlike Lawson, the Bowyer-Watson algorithm doesn't require any edge flips, but removes vertices that violate the triangulation property and reinserts those.
The algorithm can be generalized to more dimensions in a straighforward way.
This algorithm has an expected runtime of $O(n)$.
\cite{shewchuk}

\begin{algorithm}
    \caption{Bowyer-Watson}
    \begin{algorithmic}
        \Function{Triangulate}{$P$}
            \State{Intialize $T$ with a single triangle that contains all vertices of $P$}
            \ForAll{$p \in P$}
                \State{Let $D$ be the set all triangles in $T$ whose circumcircle contains $p$}
                \State{Remove triangles in $D$ from $T$}
                \State{Split the triangle that contains $p$}
                \State{Retriangulate the vertices in $D$}
            \EndFor
            \State{Remove the initial triangle \\}
            \Return $T$
        \EndFunction
    \end{algorithmic}
\end{algorithm}

\subsection{Lawson}
\label{sub:lawson}

Lawson's algorithm does use Edge-Flips.
This algorithm works in the 2-dimensional case, but is harder to generalize to more dimensions.
This algorithm has an expected runtime of $O(n)$.
\cite{shewchuk}

\begin{algorithm}
    \caption{Lawson}
    \begin{algorithmic}
        \Function{Triangulate}{$P$}
            \State{Intialize $T$ with a single triangle that contains all vertices of $P$}
            \ForAll{$p \in P$}
                \State{Locate the triangle $t$ that contains $p$ and connect the edges of $t$ to $p$ creating 3 new edges}
                \State{Call Edge-flip on $p$}
            \EndFor
            \State{Remove the initial triangle}
            \Return $T$
        \EndFunction
        \Function{Edge-flip}{$p$}
            \ForAll{triangles $t$ that contains $p$}
                \If{$p$ lies in the circumcircle of $t$}
                    \State{Flip the edge in $t$ that opposes $p$}
                    \State{Call Edge-Flip on the edges that surround $p$}
                \EndIf
            \EndFor
        \EndFunction
    \end{algorithmic}
\end{algorithm}

%A worst case analysis can be done by backward analysis.
%When we have a correct triangulation, we consider choosing a random vertex and removing that vertex from the triangulation.

%\begin{algorithm}
  %\caption{Counting mismatches between two packed \DNA strings
    %\label{alg:packed-dna-hamming}}
  %\begin{algorithmic}[1]
    %\Require{$x$ and $y$ are packed \DNA strings of equal length $n$}
    %\Statex
    %\Function{Distance}{$x, y$}
      %\Let{$z$}{$x \oplus y$} \Comment{$\oplus$: bitwise exclusive-or}
      %\Let{$\delta$}{$0$}
      %\For{$i \gets 1 \textrm{ to } n$}
        %\If{$z_i \neq 0$}
          %\Let{$\delta$}{$\delta + 1$}
        %\EndIf
      %\EndFor
      %\State \Return{$\delta$}
    %\EndFunction
  %\end{algorithmic}
%\end{algorithm}

