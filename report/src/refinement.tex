\section{Delaunay refinement}
\label{sec:refinement}

Delaunay refinement algorithms are algorithms that generate Delaunay triangulations, $D$, of a given PSLG by inserting points in its interior.
The triangles in $D$ must satisfy certain quality criterion, which serve as an input to the algorithm.
%For quality a metric can be used that is defined as the circumratio of that triangle divided by the shortest edge.
The most common criteria are the minimum angle present in each triangle and the surface area of each triangle.
These should be above resp. below a certain treshold criterion specified by the user.
Note that not all refinement algorithms can ensure termination for all tresholds.
An optimal refinement algorithm would create as few triangles as possible satisfying the specified criteria.

We will discuss two algorithms for accomplishing this task: Ruppert's refinement algorithm resp. Chew's second refinement algorithm.
The algorithms take a Delaunay triangulation satisfying the PSLG as their input and then refine it until all criterea are satisfied.
Since these algorithms insert (and possibly remove) points into the triangulation, $T$, they make use of the algorithms discussed in the previous section
to make sure that $T$ maintains the Delaunay property (and thus is equal to $D$) \cite{shewchuk}.

\subsection{Ruppert's refinement algorithm}
\label{sub:ruppert}
\nocite{shewchuk2}

In Ruppert's algorithm, two actions may occur.
Either the algorithm will split an edge, $e \in \mathrm{PSLG}$ into half,
or it will insert the circumcenter of a poor quality triangle into the triangulation.
An edge is split when its encroached by another vertex.
An edge is encroached by a vertex when the vertex is in its diametral circle. \cite{shewchuk}

Note that in each insertion step, the resulting triangulation must remain Delaunay,
which can be ensured by the algorithms of the previous section.

\begin{algorithm}
    \caption{Ruppert}
    \begin{algorithmic}
        \Function{Refinement}{PLSG}
            \State{Start with a Delaunay triangulation, $D$, satisfying the PSLG}
            \While{there is a poor quality triangle, $t$}
                \If{$t$'s circumcenter encroaches a segment}
                    \State{Split that segment into half}
                \Else
                    \State{Insert the circumcenter of $t$ in $D$}
                \EndIf
            \EndWhile
        \EndFunction
    \end{algorithmic}
    \label{alg:Ruppert}
\end{algorithm}

\subsection{Chew's second refinement algorithm}
\label{sub:chews}
\nocite{art:Rand2011}

Chew's second refinement algorithm also works by repeatedly inserting the circumcenter of a bad triangle.
Unlike Ruppert's algorithm, if the circumcenter and the triangle are separated from each other by a PSLG segment, that segment is split instead.
Also, any previously-inserted circumcenters in the diametral circle of that (unsplitted) segment are removed from the triangulation.
\cite{shewchuk}

\begin{algorithm}
    \caption{Chew}
    \begin{algorithmic}
        \Function{Refinement}{PLSG}
            \State{Triangulate the vertices bounded by the PLSG}
            \While{there is a bad triangle $t$}
                \If{$t$ is separated from its circumcenter $c$ by a segment $s$}
                    \State{Remove all previously inserted circumcenters that are in the interior of the diametral circle of $s$}
                    \State{Split $s$ in 2 subsegments of equal size}
                \Else
                    \State{Insert c in the triangulation}
                \EndIf
            \EndWhile
        \EndFunction
    \end{algorithmic}
\end{algorithm}

