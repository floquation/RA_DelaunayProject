\section{Delaunay refinement}
\label{sec:refinement}

Delaunay refinement algorithms are algorithms that generates triangulations of a polygon in of which each triangle has certain requirement on its quality and its size.
%For quality a metric can be used that is defined as the circumratio of that triangle divided by the shortest edge.
To constrain the quality, the minimum angle should be above a certain criterion.
On the other hand, we want to limit the number of triangles that gets created by the algorithm.

We will discuss two algorithms for accomplishing this task.
The first refinement algorithm is from Ruppert.
The second refinement algorithm is from Chew.
The input for these algorithms is planar straight line graph (PSLG). 
The vertices of this graph must occur in the triangulation and the segments of this graph may not be crossed.
The algorithms assume that a triangulation that satisfies these properties is already available.
To accomplish those requirements the algorithms use any of the triangulation algorithms discussed in section \ref{sec:triangulation} underneath.
\cite{shewchuk}

\subsection{Ruppert's algorithm}
\label{sub:ruppert}

In Rupperts algorithm a segment is encroached if a vertex lies inside its diametrical circle or if the segment doesn't appear in the triangulation.
As long as segments are enchroached, Rupperts algorithm splits these into subsegments.
Otherwise Rupperts algorithm inserts the circumcenter of a bad triangle.
\cite{shewchuk}

\begin{algorithm}
    \caption{Ruppert}
    \begin{algorithmic}
        \Function{Refinement}{PLSG}
            \State{triangulate the vertices bounded by PLSG}
            \While{there is a triangle $t$ that doesn't meet the requirement}
                \If{a segment is encroached}
                    \State{split that segment}
                \Else
                    \State{insert the circumcenter of $t$ in the triangulation}
                \EndIf
            \EndWhile
        \EndFunction
    \end{algorithmic}
\end{algorithm}

\subsection{Chew's second refinement algorithm}
\label{sub:chews}

Chew's algorithm also works by repeatedly inserting the circumcenter of a bad triangle.
There is one exception to this rule.
If the circumcenter and the triangle are separated from each other by a segment, that segment is split and circumcenters in the diameter of the (unsplitted) segment are removed from the triangulation.
\cite{shewchuk}

\begin{algorithm}
    \caption{Chew}
    \begin{algorithmic}
        \Function{Refinement}{PLSG}
            \State{triangulate the vertices bounded by PLSG}
            \While{there is a triangle $t$ that doesn't meet the requirement}
                \If{a triangle $t$ is separated from its circumcenter $c$ by a segment $s$}
                    \State{remove all previously inserted circumcenters that are in the interior of the diametercircle of $s$}
                    \State{split $s$}
                \Else
                    \State{insert c in the triangulation}
                \EndIf
            \EndWhile
        \EndFunction
    \end{algorithmic}
\end{algorithm}

