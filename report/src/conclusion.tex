\section{Conclusion}
\label{sec:conclusion}

We have looked at two Delaunay triangulation algorithms: Lawson's and Bowyer-Watson's.
The triangulation methods have been implemented and their performances as function of the input size has been compared
when executed on a set of $n$ randomly generated vertices.
The difference in runtime was pretty marginal, although Bowyer-Watson seemed to perform slightly faster within our experiments.

Furthermore, we have looked at Ruppert's refinement algorithm and how it performs on the boundary in the shape of a car
as a function of the termination criterea: The maximum area of each triangle and the minimum angle within each triangle.
It seemed like our implementation of Ruppert's algorithm is most likely to converge when the minimum angle was about $20\degr - 22.5\degr$.
As a function of the maximum area, there seems to be a PSLG-shape dependent complex relationship:
For certain areas the algorithm gets stuck and does not converge at all.
This seems to be purely caused by our implementation: the algorithm becomes confused (infinite loop) and has no idea what to do with certain triangles.

%I'm not sure why since intuitively I would expect that looser constraints are easier to satisfy than strict ones.
% Intuitively, looser constraints would be easier to satisfy than strict ones, so these results came as a surprise.
% The result might be specific to the structure of the boundary that we were examining.
