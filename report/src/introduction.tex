\section{Introduction}
\label{sec:introduction}

Given a set of vertices $P = \{p_1 \ldots p_n\}$, a delaunay triangulation is a triangulation $D$ of $P$
that contains edge $(u, v)$ for $u, v \in P$ if the diametercircle of $u$ and $v$ doesn't have any vertex from $P$ in its interior.

%for every edge $(u, v) \in edges(D)$ the diametercircle of $u$ and $v$ doesn't have any vertex $p \in P$ inside of it.

Another way to define the delaunay triangulation is as the dual graph of the voronoi diagram.
If we let $cell(p_i)$ be the points in $P$ that are closer to $p_i$ than to any other point of $P$,
those cells together make the voronoi diagram of $P$.

TODO: Nut van natuurkunde en refinement algorithms uitleggen :-P

We will first look at methods to find and maintain a delaunay triangulation of a set of points.
Then we will look at delaunay refinement algorithms.

% Something about applications
